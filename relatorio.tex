\documentclass[12pt,fleqn]{article}
%\usepackage {psfig,epsfig} % para incluir figuras em PostScript
\usepackage{times}
\usepackage{amsfonts,amsthm,amsopn,amssymb,latexsym}
\usepackage{graphicx}
\usepackage[T1]{fontenc}
\usepackage[brazil]{babel}
\usepackage{geometry}
\usepackage[utf8]{inputenc}
\usepackage[intlimits]{amsmath}
\usepackage{fancyhdr}
\usepackage[num]{abntex2cite}%citacao numerica, senao alf

\citebrackets[] %usando [] nas citacoes

%alguns macros
\newcommand{\R}{\ensuremath{\mathbb{R}}}
\newcommand{\Rn}{{\ensuremath{\mathbb{R}}}^{n}}
\newcommand{\Rm}{{\ensuremath{\mathbb{R}}}^{m}}
\newcommand{\Rmn}{{\ensuremath{\mathbb{R}}}^{{m}\times{n}}}
\newcommand{\contcaption}[1]{\vspace*{-0.6\baselineskip}\begin{center}#1\end{center}\vspace*{-0.6\baselineskip}}
%=======================================================================
% Dimensões da página
\usepackage{a4}                       % tamanho da página
\setlength{\textwidth}{16.0cm}        % largura do texto
\setlength{\textheight}{9.0in}        % tamanho do texto (sem head, etc)
\renewcommand{\baselinestretch}{1.15} % espaçamento entre linhas
\addtolength{\topmargin}{-1cm}        % espaço entre o head e a margem
\setlength{\oddsidemargin}{-0.1cm}    % espaço entre o texto e a margem
\setlength\parindent{24pt}     
% Ser indulgente no preenchimento das linhas
\sloppy
 
%\bibliographystyle{plain}
\begin{document}


\pagestyle {empty}

% Páginas iniciais
%\input logo

%\vspace*{-3cm}

%\begin{figure}[h]
%\leavevmode
%\begin{minipage}[t]{\textwidth}
%\includegraphics[1cm,1cm][3cm,3cm]{logo-ufrpe.bmp}
%\end{minipage}
%\end{figure}



\vspace*{-2cm}

{\bf
\begin{center}

{\large \hspace*{0cm}Universidade de Brasília} \\
\hspace*{0cm}Faculdade Gama \\
\hspace*{0cm}Engenharia Eletrônica

\end{center}}
\vspace{4.0cm}
\noindent
\begin{center}

{\Large \bf  Aplicação da tecnologia \textit{Bluetooth} para controle de acesso }\\[3cm]
{\Large Autor: Ithallo Junior Alves Guimarães}\\[6mm]
{\Large Orientadora: Dra. Lourdes Mattos Brasil}\\[3.0cm]
\end{center}






\newpage
 % capa ilustrativa

%%%%%%%%%%%%%%%%%%%%%%%%%%%%%%%%%%%%%%%%%%Folha de Rosto

\begin{titlepage}
    \vfill
    \begin{center}
        {\large Ithallo Junior Alves Guimarães} \\[5cm]
        {\Huge Aplicação da tecnologia \textit{Bluetooth} para controle de acesso}\\[1cm]
        \hspace{.35\textwidth} % posicionando a minipage
        \begin{minipage}{.5\textwidth}
            
                Dissertação apresentada na Universidade de Brasília para obtenção 
                dos créditos na disciplina de estágio supervisionado.
            
        \end{minipage}
        \vfill
        Novembro de 2017
    \end{center}
\end{titlepage}


\pagestyle {empty}
\begin{center}
\section*{\textbf Agradecimentos}

\end{center}

A esta Universidade pela oportunidade.

A minha orientadora de estágio pela prestatividade.

A empresa por me acolher.
\newpage


\tableofcontents


% Numeração em romanos para páginas iniciais (sumários, listas, etc)
%\pagenumbering {roman}
\pagestyle {plain}



\setcounter{page}{0}
\pagenumbering{arabic}
  
\setlength{\parindent}{1.25cm}  %espaco entre paragrafo e margem 
% Espaçamento entre parágrafos

\parskip 5pt

\newpage

\section{Introdução}


\hspace*{1.25cm}  O \textit{Bluetooth}  é uma tecnologia que suporta bilhões de dispositivos
 eletrônicos ao redor do mundo, seja para automação residencial ou para  aplicações médicas.
 Ele carrega a responsabilidade de ser o nome mais confiável em 
 tecnologia sem fio \cite{bluetooth}.  


Essa tecnologia é de baixo consumo de energia, quando comparada com as outras tecnologias sem fio, e  é utilizada para transferência de áudio,
dados ou mesmo para fazer transmissão de informações(\textit{Broadcast}). O \textit{Bluetooth} ainda pode ser dividido entre duas modalidades: 
BR/EDR (\textit{Basic Rate/Enhanced Data Rate}) e LE (\textit{Low Energy})  \cite{bluetooth}.


A modalidade BR/EDR permite conexões sem fio contínuas de um para um (1:1), 
utilizando conexões de ponta a ponta (P2P). Ela é ideal para transmissão 
de áudio. Já a  LE (BTLE) pode ser utilizado tanto para conexões P2P (um para um, ideal para transferência de dados, como monitores de saúde), 
transmissão de informações (\textit{Broadcast}) na forma de um 
para muitos (como em \textit{beacons}) ou mesmo em redes \textit{mesh} (muitos para muitos), 
usada em automação e redes de sensores \cite{bluetooth}.

Por ser tão presente no dia a dia de boa parte de população a ponto de ser esquecida ou tomado por certo, o \textit{Bluetooth} juntamente
com os \textit{smartphones} se tornam tecnologias extremamente atraentes ao ramo do controle de dispositivos e internet das coisas.

\section{Descrição da Empresa}   


\hspace*{1.25cm} A Loop Engenharia da Computação \cite{loop} é uma  \textit{startup} natural de Brasília  dedicada a 
construção de \textit{hardware} e  \textit{software},  que atua desde de 2011 no mercado e 
tem como principal produto o LoopKey, uma solução de acesso.


O LoopKey elimina a necessidade de se andar com chaves e permite o compartilhamento rápido e prático
de acesso a portas, bem como controle inteligente dos horários de acesso.

Assim, apesar de ser uma empresa relativamente nova, ela conta com pessoal altamente capacitado, bem como com eficientes times de
engenheiros para desenvolvimento de \textit{software} e \textit{hardware}.


A empresa atualmente se encontra em aceleração no Centro de Desenvolvimento Tecnológico da Universidade de Brasília (CDT-UnB),
localiza-se na Asa Norte e possui escritórios tanto em Brasília quanto em São Paulo.

\section{Atividades Desenvolvidas e Cronogramas de Execução}

\hspace*{1.25cm} A Tabela \ref{cron} mostra o cronograma de atividades proposto inicialmente, no que diz respeito aos meses e 
ao que seria feito.


\begin{table}[h!]
\centering
\caption {Cronograma de atividades proposto.}

\label{cron}
\resizebox{\textwidth}{!}{%
\begin{tabular}{| c || c |} 
\hline
Mês &  Atividade \\ 
\hline
1º & Capacitação em tecnologias de automação residencial \\ 
6º & Implementação de sistemas de controle de acesso \\
6º & Desenvolvimento de metodologias de processos para projetos \\
6º & Aprendizado de em projeto de circuitos eletrônicos para radiofrequência com tecnologia \textit{Bluetooth} \\
\hline
\end{tabular} %
}
\end{table}

No primeiro mês, conseguiu-se fazer um estudo de automação residencial, na parte que tange a automação de portas para
controle inteligente via \textit{smartphone}. Isso se deu por meio do estudo de como funciona a comunição via 
\textit{Bluetooth}, bem como os protocolos internos de comunicação da empresa.

Após a compreensão do funcionamento da tecnologia, o estagiário teve de implementar um sistema de comunicação UART
( \textit{universal asynchronous receiver-trasmitter}) via \textit{Bluetooth}. Esse sistema seria posteriormente utilizado
no controle de acesso para a empresa. 

Para tanto, houve um estudo sobre sistemas embarcados,  sistemas operacionais bem como
de bibliotecas para melhor suprir as necessidades do sistema. A linguagem de programação \textit{Python} foi escolhida
devido a sua praticidade, documentação e relativa facilidade  e velocidade
para se fazer prótipos de teste/aplicações finais.

A elaboração do sistema de controle de acesso se deu por meio do uso de placas de controle de acesso proprietárias da empresa (LoopKey),
dispositivos \textit{Bluetooth} rodando em  sistemas embarcados Linux e programação em \textit{Python}, sendo o estagiário
responsável pela preparação do sistema embarcado e sua configuração integral. 

Esse sistema foi utilizado para teste de novos dispositivos (testes de funcionamento, distância de operação e tempo 
de resposta).

Durante todo esse processo, o estagiário teve de se colocar por dentro das metodologias de projetos e formas de autogestão,
para assim poder executar de forma eficiente e dentro dos prazos o que lhe foi proposto.

O tempo gasto na empresa permitiu ao estagiário compreender o papel e a responsabilidade de um engenheiro no mundo real, lidando
com prazos, consequências, interagindo de forma profissional com seus colegas de trabalho,
bem como compreendendo a estrutura organizacional interna de empresas.

\newpage
\section{Resultados e Discussão}

 \hspace*{1.25cm} Sistemas de controle de acesso foram desenvolvidos pelo estagiário. 
 Deste modo, o mesmo trabalhou integrando sistemas embarcados Linux (e.g. \textit{Raspberry pi}), 
 sistemas microcontrolados e a tecnologia \textit{Bluetooth}. 

 O que foi desenvolvido encontra-se atualmente em fase de implementação em um novo produto, que integra a solução 
 de acesso já fornecida pela empresa com a possibilidade de uso de múltiplos dispositivos ao mesmo tempo. 

\section{Considerações Finais}
\hspace*{1.25cm} Ainda há atividades a serem executadas, uma vez que o prazo do estágio 
permite isso (seis meses). Deste modo, ainda há a possibilidade de se aprender muito além do planejado e adquirir experiências
muito relevantes tanto para o mercado de trabalho quanto para o papel de engenheiro na sociedade.


\addcontentsline{toc}{section}{Referências}
%\bibliographystyle{abnt-alf}
\bibliography{ref}
%\section{Anexos}


\end{document}
